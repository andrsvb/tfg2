\documentclass[a4paper]{article}

\usepackage[utf8]{inputenc}
\usepackage[spanish]{babel}
\usepackage{color}
\usepackage{amsmath,amssymb}
\usepackage[python3]{texsurgery}

%%%%%% ---header

\title{Test casillas marcadas}
\author{andres}

\begin{document}


%%% --- pregunta vf1
\element{cat1}{
\begin{question}{v_f1}
Si \(a\) es una solución para una ecuación, entonces \((x + a)\) es un factor de la ecuación. 
\begin{choices}
    \wrongchoice{Verdadero}
    \correctchoice{Falso}
\end{choices}
\end{question}
}

%%% --- pregunta vf2
\element{cat2}{
\begin{question}{v_f2}
El coeficiente de \(5x^2\) es \(2\).
\begin{choices}
    \wrongchoice{Verdadero}
    \correctchoice{Falso}
\end{choices}
\end{question}
}

%%% --- pregunta vf3
\element{cat3}{
\begin{question}{v_f3}
Los ángulos son funciones de los lados
\begin{choices}
    \wrongchoice{Verdadero}
    \correctchoice{Falso}
\end{choices}
\end{question}
}

%%% --- pregunta vf4
\element{cat4}{
\begin{question}{v_f4}
Todo triángulo acutángulo tiene sus tres lados de distinta longitud
\begin{choices}
    \wrongchoice{Verdadero}
    \correctchoice{Falso}
\end{choices}
\end{question}
}

%%% --- pregunta vf5
\element{cat5}{
\begin{question}{v_f5}
 La suma de los ángulos interiores de un hexágono es de 720º
 \begin{choices}
    \correctchoice{Verdadero}
    \wrongchoice{Falso}
\end{choices}
\end{question}
}

%%% --- pregunta vf6
\element{cat6}{
\begin{question}{v_f6}
El semiperímetro de un triángulo equilátero con base \(b=5\) cm es \(s= 7.5 \) cm. 
\begin{choices}
    \correctchoice{Verdadero}
    \wrongchoice{Falso}
\end{choices}
\end{question}
}


%%% --- pregunta mul1
\element{cat7}{
\begin{question}{mul1}
¿Cuál de los siguientes números es irracional? 
\begin{choices}
        \wrongchoice{$\sqrt{4000}$}
        \wrongchoice{$-\sqrt{6}$} 
        \wrongchoice{$\sqrt{2}$}
        \correctchoice{$17$}
        \end{choices}
\end{question}
}

%%% --- pregunta mul2
\element{cat8}{
\begin{question}{mul2}
Resuelve la ecuación: \( \left|  5-x  \right| =5\)
    \begin{choices}
            \correctchoice{$0, 10$}
            \wrongchoice{$0, 5$}
            \wrongchoice{$-5, 10$}
            \wrongchoice{$-10, 0$}
    \end{choices}
\end{question}
}

%%% --- pregunta mul3
\element{cat9}{
\begin{question}{mul3}
    Resuelve la ecuación: \(\left| 5x+5 \right|= 10 \)
    \begin{choices}
            \correctchoice {$-3, {1}$}
            \wrongchoice {$-3, {-1}$}
            \wrongchoice {$3, {1}$}
            \wrongchoice {$3, {-1}$}
    \end{choices}
\end{question}
}

%%% --- pregunta mul4
\element{cat10}{
\begin{question}{mul4}
     Resuelve por factorización:  \(x^2 - 2 x - 8  =0. \  \ x = \)
        \begin{choices}
            \correctchoice {\(-2, 4\)}
            \wrongchoice {\( -2, 2\)}
            \wrongchoice {\(-2, 0 \)}
            \wrongchoice {\(2, -4\)}
        \end{choices}
\end{question}
}

%%% --- pregunta mul5
\element{cat11}{
\begin{question}{mul5}    
Halla los valores de \(x\) si \   \((x+3)^2 = 100\)
        \begin{choices}
            \correctchoice {$7, -13$}
            \wrongchoice {$-7,-13$}
            \wrongchoice {$-7, 13$}
        \end{choices}        
\end{question}
}

%%% --- pregunta mul6
\element{cat12}{
\begin{question}{mul6}
        Halla \(x\) si \ \( x^2 - x = 6  \).        
   \begin{choices}
            \correctchoice {$-2, 3$}
            \wrongchoice {$-3, 2$}
            \wrongchoice {$2, 3$}
            \wrongchoice {$6, -1$}
\end{choices}        
\end{question}
}

%%% --- pregunta mul7
\element{cat13}{
\begin{question}{mul7}
        Completa el cuadrado para hallar las soluciones a $\ x(x+4) = 0 $        
   \begin{choices}
            \wrongchoice {$0, 2$}
            \wrongchoice {$-2, 2$}
            \wrongchoice {$-2, 0$}
            \correctchoice {$-4,0$}
   \end{choices}        
\end{question}
}

%%% --- pregunta mul8
\element{cat14}{
\begin{question}{mul8}
        Construye un polinomio de segundo grado a partir de las soluciones \(x=2, x=1\).
        \begin{choices}
            \wrongchoice {$x^2 - 3x + 2$}
            \correctchoice {$x^2 + 3x + 2$}
            \wrongchoice {$x^2 +2x + 2$}
            \wrongchoice {$x^2 +2$}
        \end{choices}        
\end{question}
}

%%% --- pregunta mul9
\element{cat15}{
\begin{question}{mul9}        
         Expresa en forma estándar \((x-h)^2 = k\) la ecuación   \ \ \(x^2 + 8x+7= 27\)         
    \begin{choices}
            \wrongchoice {$(x-4)^2=16$}
            \wrongchoice {$(x-4)^2=13$}
            \wrongchoice {$(x+4)^2=-36$}
            \correctchoice {$(x+4)^2=36$}
   \end{choices}        
\end{question}
}

%%% --- pregunta mul10
\element{cat16}{
\begin{question}{mul10}
            Factoriza:  \ \ \(x^2 -2x +1 \)
        \begin{choices}
            \wrongchoice {$(x+2)(x -1)$}
            \correctchoice {$(x-1)(x-1)$}
            \wrongchoice {$(x-2)(x-1)$}
            \wrongchoice {$x(x+1)$}
        \end{choices}         
\end{question}
}

%%% --- pregunta mul11
\element{cat17}{
\begin{question}{mul11}
        Expresa en forma estándar \((x-h)^2 = k\) la ecuación : \ \ \(x^2 + 6x +4 = 0\)        
 \begin{choices}
            \correctchoice {$(x+3)^2 = 5$}
            \wrongchoice {$(x-3)^2 = -13$}
            \wrongchoice {$(x+3)^2 = 13$}
            \wrongchoice {$(x-3)^2 = -5$}
\end{choices}        
\end{question}
}

%%% --- pregunta mul12
\element{cat18}{
\begin{question}{mul12}         
            Factoriza:  \ \ \(x^2+x -2 \)
        \begin{choices}
            \correctchoice {$(x+2)(x -1)$}
            \wrongchoice {$(x-1)(x-1)$}
            \wrongchoice {$(x-2)(x-1)$}
            \wrongchoice {$(x+1)(x-2)$}
        \end{choices}        
\end{question}
}

%%% --- pregunta mul13
\element{cat19}{
\begin{question}{mul13}
La suma resultante de dos números, \(a\) y \(b\), es \(16\). Si su producto es \(63\), encuentra\(a\) y \(b\):
    \begin{choices}
         \wrongchoice {\(7, 3.5\)}
         \wrongchoice {\(4,7\)}
         \correctchoice {\(7, 9\)}
         \wrongchoice {\(9, 5\)}
    \end{choices}
\end{question}
}

%%% --- pregunta mul14
\element{cat20}{
\begin{question}{mul14}
 Halla el valor de cada ángulo interno de un pentágono equilatero.
        \begin{choices}
            \correctchoice {108º}
            \wrongchoice {72º}
            \wrongchoice {110º}
            \wrongchoice {540º}
        \end{choices}
\end{question}
}

% #################################################################
% C R E A T I O N  D E S  C O P I E S
% #################################################################
\exemplaire{1}{    	% nombre de sujet différent
  %debut de l'en-tête des copies :
  \vspace*{.5cm}
  \begin{minipage}{.4\linewidth}
    \centering\large\bf Test
  \end{minipage}
  \champnom{\fbox{
      \begin{minipage}{.5\linewidth}
        Nom et prénom :
        \vspace*{.5cm}\dotfill
        \vspace*{1mm}
      \end{minipage}
    }}

  \begin{flushleft}
Test exam to check different ways to check boxes and to test exams with multiple pages
\begin{center}
Primavera 2018 \\
\textbf{Examen Extraordinario de Matemáticas II } \\
2do Semestre \\
Salón 41
\end{center}
\textbf{Objetivo:} Evaluar las competencias descritas en el programa analítico referentes a la resolución de ecuaciones cuadráticas, identificación y uso de elementos geométricos y uso de la trigonometría en problemas aplicados a contextos reales.


\parbox{6in}{{\textbf{Instrucciones Generales:} Lee tu examen cuidadosamente. Pon atención a los detalles. Llena cuidadosamente tu hoja de respuestas ( no olvides marcar también la respuesta en tu examen). Asegúrate que tu calculadora esté en modo \textit{DEG}, y no \textit{RAD}. }
}

\fbox{\fbox{\parbox{6in}{
\begin{itemize}
    \item Tienes 120 minutos para contestar este examen.
    \item Puedes utilizar calculadora y formulario. No puedes utilizar ningún otro tipo de dispositivo como calculadora. Si no tienes calculadora, deberás esperar a que alguien algún compañero termine y entregue su examen. 
    \item Cuando termines, no olvides entregar tanto tu examen como tu  \textit{scantron}. 
\end{itemize} }}}

  \end{flushleft}


  % mélange et catégorie (groupe dans AMC)
  \cleargroup{BigGroupe}
  \copygroup{cat1}{BigGroupe}
  \copygroup{cat2}{BigGroupe}
  \copygroup{cat3}{BigGroupe}
  \copygroup{cat4}{BigGroupe}
  \copygroup{cat5}{BigGroupe}
  \copygroup{cat6}{BigGroupe}
  \copygroup{cat7}{BigGroupe}
  \copygroup{cat8}{BigGroupe}
  \copygroup{cat9}{BigGroupe}
  \copygroup{cat10}{BigGroupe}
  \copygroup{cat11}{BigGroupe}
  \copygroup{cat12}{BigGroupe}
  \copygroup{cat13}{BigGroupe}
  \copygroup{cat14}{BigGroupe}
  \copygroup{cat15}{BigGroupe}
  \copygroup{cat16}{BigGroupe}
  \copygroup{cat17}{BigGroupe}
  \copygroup{cat18}{BigGroupe}
  \copygroup{cat19}{BigGroupe}
  \copygroup{cat20}{BigGroupe}
  % not usefull for testing !
  %\melangegroupe{BigGroupe}
  \restituegroupe{BigGroupe}
}

\end{document}
  
