\documentclass[11pt,a4paper]{article}
\usepackage[utf8]{inputenc}

\usepackage{a4wide}

\usepackage{amsmath,amssymb,mathrsfs,amsthm,mathtools} % Matemáticas varias


% --- Para el encabezado
\usepackage{fancyhdr}
\fancyhead[L]{\textsf{Control XX} (XX/XX/XX)}\fancyhead[R]{\textsf{Cálculo I -- Escuela de Navales -- UPM}}\fancyfoot[C]{\thepage}
\pagestyle{fancy}


% --- Mates
\DeclareMathOperator{\sen}{sen}

% -- Ambientes Ejercicio y Solucion
\newtheorem{ejercicio}{Ejercicio}
\theoremstyle{definition}
\newtheorem{solucion}{Soluci\'on}


\begin{document}






% =================================================================
% ENUNCIADO
% =================================================================
\begin{ejercicio}
  Aqui va el enunciado...por partes:
  \begin{enumerate}
    \item Parte uno....realizar con las siguientes condiciones:
    \begin{itemize}
      \item no debe hacer tal
      
      \item pero si taltal
      
      \item
    \end{itemize}
    
    \item Parte dos....realizar por orden:
    \begin{enumerate}
      \item tira lo primero
      
      \item tira por lo segundo
      
      \item ...
    \end{enumerate}
  \end{enumerate}
\end{ejercicio}

\bigskip




% =================================================================
% SOLUCION
% % =================================================================
\begin{solucion}
  Aqui va la solucion....
\end{solucion}



\end{document}
