\documentclass[11pt,a4paper]{article}
\usepackage[utf8]{inputenc}

\usepackage[spanish]{babel}
\usepackage{tcolorbox}
\usepackage[python3]{texsurgery}

\usepackage{a4wide}

\usepackage{amsmath,amssymb,mathrsfs,amsthm,mathtools} % Matemáticas varias


% --- Para el encabezado
%\usepackage{fancyhdr}
%\fancyhead[L]{\textsf{Control XX} (XX/XX/XX)}\fancyhead[R]{\textsf{Cálculo I -- Escuela de Navales -- UPM}}\fancyfoot[C]{\thepage}
%\pagestyle{fancy}


\title{Examen de prueba pyexams}
\author{Andres Bartolome}

\makeatletter
\newcommand{\AffichageSiCorrige}[1]{\ifAMC@correc #1\newline \fi}
\makeatother

\usepackage{listings}
\lstset{
upquote=true,
columns=fullflexible,
basicstyle=\ttfamily,
}

% --- Mates
%\DeclareMathOperator{\sen}{sen}

% -- Ambientes Ejercicio y Solucion
\theoremstyle{definition}
\newtheorem*{ejercicio}{Ejercicio}
\newtheorem*{solucion}{Soluci\'on}


%#########################################################################
% Document
%#########################################################################
\begin{document}


\begin{runsilent}
from random import seed, randint
import numpy as np
seed(\seed)
np.random.seed(\seed)
\end{runsilent}


% =================================================================
% ENUNCIADO
% =================================================================
\begin{question}{pregunta}
\begin{ejercicio}

\begin{runsilent}
sin_cos = randint(0,1)
\end{runsilent}

    ecuación con funciones complejas \\
    \begin{tcolorbox}[colback=blue!10!white,colframe=blue!90!black,fonttitle=\bfseries,title=MOODLE]
        Hallar todos los $z \in \mathbb C$ verificando:
        \begin{equation*}
            \sif{sin_cos}{(1-2i)\sin{z}=-\frac{5}{3}+\frac{10}{3}i}{(-1+2i)\cos{z}=\frac{8}{3}+\frac{4}{3}i}
        \end{equation*}
        (La respuesta se ha de dar de forma exacta. Detallar la respuesta y los c\'alculos que llevan a ella EN UN SOLO PDF que se ha de adjuntar a esta pregunta. En el mismo PDF debe aparecer tanto el nombre y apellidos del alumno/a así como un documento de identidad.)
    \end{tcolorbox}
\end{ejercicio}

\bigskip




% =================================================================
% SOLUCION
% % =================================================================

\end{question}
\AffichageSiCorrige{ 	
%\begin{solucion}
  Aplicando la definición, expresamos la función trigonométrica compleja en función de exponenciales complejas:
  \begin{equation*}
      \sif{sin_cos}{\sin{z}=\frac{e^{iz}-e^{-iz}}{2i}}{\cos{z}=\frac{e^{iz}+e^{-iz}}{2}}
  \end{equation*}
  Sustituyendo en la equación original, obtenemos
  \begin{equation*}
      \sif{sin_cos}{(1-2i)\frac{e^{iz}-e^{-iz}}{2i}=-\frac{5}{3}+\frac{10}{3}i}{(-1+2i)\frac{e^{iz}+e^{-iz}}{2}=\frac{8}{3}+\frac{4}{3}i}
  \end{equation*}
  Operando, obtenemos la siguiente igualdad
  \begin{equation*}
  \begin{split}
      \sif{sin_cos}{e^{iz}-e^{-iz}&=\frac{(-\frac{5}{3}+\frac{10}{3}i)2i}{1-2i}\\&=\frac{(-\frac{20}{3}-\frac{10}{3}i)(1+2i)}{(1-2i)(1+2i)}\\&=\frac{-\frac{50}{3}i}{5}\\&=-\frac{10}{3}i}{e^{iz}+e^{-iz}&=\frac{(\frac{8}{3}+\frac{4}{3}i)2}{-1+2i}\\&=\frac{(\frac{16}{3}+\frac{8}{3}i)(-1-2i)}{(-1+2i)(-1-2i)}\\&=\frac{-\frac{40}{3}i}{5}\\&=-\frac{8}{3}i}
  \end{split}
  \end{equation*}
  Por lo que
  \begin{equation*}
      \sif{sin_cos}{e^{iz}-e^{-iz}-(-\frac{10}{3}i)=0}{e^{iz}+e^{-iz}-(-\frac{8}{3}i)=0}
  \end{equation*}
  Multiplicando todo por $e^{iz}$, nos queda:
  \begin{equation*}
      \sif{sin_cos}{e^{2iz}+\frac{10}{3}ie^{-iz}-1=0}{e^{2iz}+\frac{8}{3}ie^{-iz}+1=0}
  \end{equation*}
  Ahora, haciendo el cambio de variable $w=e^{iz}$, lo anterior es equivalente a obtener las soluciones del polinomio de grado 2:
  \begin{equation*}
      \sif{sin_cos}{w^2+\frac{10}{3}iw-1=0}{w^2+\frac{8}{3}iw+1=0}
  \end{equation*}
  Las raíces son:
  \begin{equation*}
      \sif{sin_cos}{w=\frac{-\frac{10}{3}i \pm \sqrt{(-\frac{10}{3}i)^2-4(-1)}}{2}=\frac{-\frac{10}{3}i \pm \sqrt{-\frac{64}{9}}}{2}=\frac{-\frac{10}{3}i \pm \frac{8}{3}i}{2}=-\frac{5}{3}i \pm \frac{4}{3}i}{w=\frac{-\frac{8}{3}i \pm \sqrt{(-\frac{8}{3}i)^2-4(1)}}{2}=\frac{-\frac{8}{3}i \pm \sqrt{-\frac{100}{9}}}{2}=\frac{-\frac{8}{3}i \pm \frac{10}{3}i}{2}=-\frac{4}{3}i \pm \frac{5}{3}i}
  \end{equation*}
  Ahora, para cada raíz
  \begin{align*}
      \sif{sin_cos}{w_1&=-\frac{1}{3}i & &y & w_2&=-3i}{w_1&=\frac{1}{3}i & &y & w_2&=-3i}
  \end{align*}
  debemos resolver la correspondiente ecuación $e^{iz}=w_j$, para deshacer el cambio de variables en cada caso y obtener todas las familias de soluciones.
  
 -\underline{$e^{iz}=w_1$}: como \sif{sin_cos}{$-\frac{1}{3}i=(\frac{1}{3})e^{-\frac{\pi}{2}i}$}{$\frac{1}{3}i=(\frac{1}{3})e^{\frac{\pi}{2}i}$}, tomando logaritmos, obtenemos:
      \begin{align*}
          \sif{sin_cos}{iz=\ln{\frac{1}{3}}&+i(-\frac{\pi}{2}+2\pi k)\\\Longrightarrow z=-i\ln{\frac{1}{3}}+(-\frac{\pi}{2}+2\pi k)&\Longrightarrow \fbox{$z=i\ln{3}+(\frac{\pi}{2}+2\pi k)$, \: con $k \in \mathbb Z$}}{iz=\ln{\frac{1}{3}}&+i(\frac{\pi}{2}+2\pi k)\\\Longrightarrow z=-i\ln{\frac{1}{3}}+(\frac{\pi}{2}+2\pi k)&\Longrightarrow \fbox{$z=i\ln{3}+(\frac{\pi}{2}+2\pi k)$, \: con $k \in \mathbb Z$}}
      \end{align*}
  - \underline{$e^{iz}=w_2$}: análogamente $-3i=(3)e^{-\frac{\pi}{2}i}$. Tomando logaritmos de nuevo:
      \begin{align*}
          iz=\ln{3}&+i(-\frac{\pi}{2}+2\pi k)\\
          \Longrightarrow z=-i\ln{3}+(-\frac{\pi}{2}+2\pi k)
          &\Longrightarrow \fbox{$z=i\ln{\frac{1}{3}}+(-\frac{\pi}{2}+2\pi k)$, \: con $k \in \mathbb Z$}
      \end{align*}  
%\end{solucion}
}

\end{document}

